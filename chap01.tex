\chapter{Abstract Interpretation}

\xxx{introduction}

\section{Introductory Example} %=======================================================================================%

\xxx{introductory example}

\section{Lattices} %===================================================================================================%

Recall the definition of partial order:

\begin{defn}[Partial order]
    Relation $\leq$ on set $S$ is an partial order if $\forall a, b, c \in P$:
    \begin{enumerate}
        \item $a \leq a$ (Reflexivity)
        \item $a \leq b \land b \leq a \implies a = b$ (Antisymmetry)
        \item $a \leq b \land b \leq c \implies a \leq c$ (Transitivity)
    \end{enumerate}
\end{defn}

Then defining a partially ordered set (Poset) is straightforward:

\begin{defn}[Poset]
    The pair $(S, \leq)$ is a Poset, if $\leq$ is a partial order on $S$
\end{defn}

We will also mention few examples of Poset:

\begin{itemize}
    \item $(\mathbb{R}, \leq), (\mathbb{Q}, \leq), (\mathbb{Z}, \leq)$ are all Posets
    \item For a set $S$, $(\mathbb{P}(S), \subseteq)$ is a Poset
    \item For directed acyclic graph $G=(V,E)$ the pair $(V, reachability)$ is a Poset
\end{itemize}

Last thing that we need before defining a Lattice are the definitions of supremum and infimum.

\begin{defn}
    On partially ordered set $(S, \leq)$, for $R \subseteq S$:

    Upper bound of $R$ in $S$ is $a \in S$ such that $\forall x \in R: x \leq a$

    Lower bound of $R$ in $S$ is $a \in S$ such that $\forall x \in R: a \leq x$

    Supremum of $R$ in $S$ is $a \in S$ such that $a$ is a lower bound of $R$

    Infimum of $R$ in $S$ is $a \in S$ such that
\end{defn}

With this knowledge, we have all we need to define a Lattice:
\begin{defn}[Lattice]

\end{defn}


\xxx{lattices}



\section{Galois Connections} %=========================================================================================%

\xxx{galois connections}

\section{The method} %=================================================================================================%

\section*{Summary}
\addcontentsline{toc}{section}{Summary}