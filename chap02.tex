\chapter{Data manipulation and Pandas library}

In this chapter, our goal is to explore the world of data manipulation.
We explain which data structures are usually used and what do the operations on them look like.
Then we explain how Pandas, Python library for data manipulation, helps with the problem of data manipulation.
We will need the information in next chapters when we define the Abstract Interpretation framework for these data
structures.
At the end of the chapter, we again mention the option based on type-systems.
However, it is rather a brief overview than a deep dive.


\section{Data structures} %============================================================================================%

When we talk about data structure, we usually define it as a way of organizing data in the computer memory.
However, there are two concepts to distinguish - the interface and the implementation.

The interface is a set of operations that we are allowed to do with the data structure as a users.
Good example of a data structure interface can be an Array, List, Dictionary or a Heap.
Implementation on the other hand is how the data structure works under the hood to provide the interface promised.
To give an example of an implementation of a data structure I mention a binary tree, n-ary heap or a linked list.

In this chapter, when talking about data structures, we omit the implementation details, and we focus only on the
interface of the data structure i.e.\ what operations are provided.
Also, we assume existence of the primitive data types such as integers, floating-point numbers, strings etc.


\subsection{Series}

The first and the simplest data structure is usually called Series.
In its simplest form it is a one-dimensional data structure holding data of some primitive data type.
It supports a List interface, so random access based on integer indexing is supported as well as adding, removing and
modifying items.

There can also be an index associated with the Series.
The index is a set of distinct values of any elementary type (usually a string or a time) associated with the values of
the Series.
It expands the interface of a Series with a Dictionary interface.
Consequently, the items can be accessed using the values in the index. \xxx{TODO a picture would be nice}

Another usual feature of a Series is an optional label describing the data. \xxx{TODO explain better}



\subsection{Dataframe}

Dataframe is a two-dimensional tabular data structure.
A good way to look at a Dataframe is to see it as a Dictionary, where the keys are names of a columns of a table
and values are Series representing the columns themselves.
This also means that the Dataframe supports indexing of columns based on column names and integer-based indexing of rows.

The dataframe can alo have an index associated with the rows of Series.
Consequently, the Dataframe supports Dictionary interface on both rows and columns.
Adding of new columns and rows is also supported.
\xxx{TODO picture}


\section{Common operations} %==========================================================================================%

To actually manipulate the data into some useful form, we need operations that more powerful than just indexing and
adding and removing elements.
The operations that are commonly used in data manipulation were greatly influenced byt the operations on relational
databases and the SQL language.
However, they are usually more flexible and customizable. % todo explain more

% indexing (sub, one...), merge, groupby, sort, renaming, vectorized operations (sum, product)
\subsection{SQL-inspired operations}


\subsection{Vectorized operations}


\section{Pandas approach} %============================================================================================%


\section{Type-based safety options} %==================================================================================%


\section*{Summary} %===================================================================================================%
\addcontentsline{toc}{section}{Summary}

We covered two common data structure interfaces used in data analysis - Series and Dataframe.
Series is a one-dimensional List-like data structure with support for index and an axis label.
Data frame is a two-dimensional