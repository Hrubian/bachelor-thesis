\chapter{User documentation}\label{ch:user-documentation}


\section{Building from source}

To build the Pandalyzer from sources, follow the steps below:
\begin{enumerate}
    \item Ensure that you have Java (version 21.0.1 or higher), Git and Python 3.x installed.
    \item Clone the Pandalyzer repository:

    \verb|git clone https://github.com/Hrubian/Pandalyzer.git|
    \item Navigate to the root folder of the repository:

    \verb|cd Pandalyzer|
    \item Run the Gradle bootstrap script:

    \verb|./gradlew build| or \verb|./gradlew.bat build| on Windows
\end{enumerate}

\section{Running the tool}

The build generates a \verb|./build/| folder.
Check that there are also \\ \verb|./build/distributions/Pandalyzer.tar| \\ and
\verb|./build/distributions/Pandalyzer.zip| archives.
Unpack one of them (depending on what tools you are provided with) and run the \verb|Pandalyzer| (or
\verb|Pandalyzer.bat|) script in the \verb|bin| folder.
The program accepts the following command-line arguments:
\begin{itemize}
    \item \verb|-h, --help| - Prints usage information and exits
    \item \verb|-i, --input <arg>| - The input python script to analyze \textbf{(mandatory)}
    \item \verb|-o, --output <arg>| - The output file to store the analysis result to (standard output by default)
    \item \verb|-c, --config <arg>| - The configuration file to read the file structures from (config.toml by default)
    \item \verb|-f, --format <arg>| - The format of the analysis output, possible options: hr (human-readable), json (hr by default), csv
\end{itemize}
