\chapter{User documentation}\label{ch:user-documentation}


\section{Building from source}

To build the Pandalyzer from sources, follow the steps below:
\begin{enumerate}
    \item Ensure that you have JDK (version \xxx{todo version}), Gradle (version \xxx{todo version}), Git
    and Python 3.x installed.
    \item Clone the Pandalyzer repository:

    \verb|git clone https://github.com/Hrubian/Pandalyzer.git|
    \item Navigate to the root folder of the repository:

    \verb|cd Pandalyzer|
    \item Run the Gradle bootstrap script:

    \verb|./gradlew build| or \verb|./gradlew.bat build| on Windows
\end{enumerate}

\section{Running the tool}

The build generates a .jar file \xxx{TODO where}.
This file can be run with the following command:
\verb|java --jar \xxx{todo file path}|.
The program accepts the following command-line arguments:
\begin{itemize}
    \item -h, --help - Prints usage information and exits
    \item -i, --input - The input python script to analyze \textbf{(mandatory)}
    \item -o, --output - The output file to store the analysis result to (standard output by default)
    \item -c, --config - The configuration file to read the file structures from (config.toml by default)
    \item -f, --format - The format of the analysis output, possible options: hr (human-readable), json (hr by default), csv
\end{itemize}

The behaviour and output of the program will be discussed in the next chapter.
