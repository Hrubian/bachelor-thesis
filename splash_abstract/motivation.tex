\section{Motivation}

Pandas~\cite{pandas_docs} is a widely used Python library for manipulating tabular data,
also known as \emph{data frames}.
The dynamic nature of Pandas and Python can be a source of runtime errors such as misspelled column names, nonexistent
columns, incompatible column types, and other invalid operations.
Consider the script shown in the Listing~\ref{lst:motivation_script}
that uses the Titanic dataset from the Pandas tutorial~\cite{pandas_statistics_2020}.

\definecolor{mygray}{rgb}{0.5,0.5,0.5}
\definecolor{mymauve}{rgb}{0.58,0,0.82}

\lstset{ %
    backgroundcolor=\color{white},   % choose the background color
    basicstyle=\footnotesize,        % size of fonts used for the code
    breaklines=false,                 % automatic line breaking only at whitespace
    captionpos=b,                    % sets the caption-position to bottom
    commentstyle=\color{red},    % comment style
    escapeinside={\%*}{*)}, % if you want to add LaTeX within your code
keywordstyle=\color{blue},       % keyword style
stringstyle=\color{mymauve},     % string literal style
}
\lstinputlisting[
    basicstyle=\footnotesize\ttfamily,
    language=Python,
    label={lst:motivation_script},
    caption=Example Pandas code with mistakes,
    captionpos=b,
    numbers=left,
]{splash_abstract/listings/lst_motivation.py}

Our goal is to propose a framework for the analysis of data manipulation programs and design a tool
leveraging the proposed framework.
The tool should be capable of detecting common Pandas errors similar to the ones in the Listing~\ref{lst:motivation_script}.

