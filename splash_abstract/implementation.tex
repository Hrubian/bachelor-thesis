\section{Implementation}

Using the abstract domain outlined in the previous section, I implemented a static checker for Pandas called Pandalyzer.
It is implemented in Kotlin~\cite{kotlinDocs} and can be found on GitHub~\cite{pandalyzer}.
Pandalyzer has a command-line interface and can also produce JSON output, suitable for use in continuous integration.

The tool first loads the Python script from the given input file.
Then it parses the code and creates its abstract syntax tree (AST).
Then it goes over the statements in the body of the module and interprets them while keeping the current context
containing all currently active variables, raised errors and warnings, etc.
The interpreter calls our abstract implementation of standard pandas library methods, which operates on the structures
of the abstract lattice and computes appropriate abstract structure as the result.
Then it outputs the analysis result to the standard output or a file.

Pandalyzer uses the fork-join pattern to interpret if-statement, where we are unable to derive which
branch would be taken and combines the results to a NondeterministicStructure.
The result of an invalid operation is the UnresolvedStructure, which ensures that the Pandalyzer is able to continue
with the interpretation even when an error in the code is discovered.
Pandalyzer obtains information about the input csv file via a configuration file.
The configuration file also supports regex filenames.