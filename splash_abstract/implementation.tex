\section{Implementation}

The tool is implemented in Kotlin, and the source code can be found in the Pandalyzer GitHub repository (TODO cite).
The tool has a command-line interface and produces (apart from human-readable) also JSON output, which makes it well
suited for automatic(TODO Explain pipelines and CI).

Let us go over what the tool does.
It loads the Python script from the given input file.
Then it parses the code and creates and abstract syntax tree (AST) of the module.
Then it goes over the statements in the body of the module and interprets them while keeping the current context
containing all currently active variables, raised errors and warnings, etc.
Finally, it writes the result of the analysis to the standard output or to an output file (if provided).

TODO uncertainty, invalid state, uses the framework\ldots