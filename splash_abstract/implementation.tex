\section{Implementation}

Using the abstract domain outlined in the previous section, I implemented a static checker for Pandas called Pandalyzer.
It is implemented in Kotlin~\cite{kotlinDocs}, and the source code can be found in the Pandalyzer GitHub
repository~\cite{pandalyzer}.
The tool has a command-line interface and produces (apart from human-readable) also JSON output,
which also makes it well suited for automatic testing (for example, in GitLab pipelines).

The tool first loads the Python script from the given input file.
Then it parses the code and creates an abstract syntax tree (AST) of the module.
Then it goes over the statements in the body of the module and interprets them while keeping the current context
containing all currently active variables, raised errors and warnings, etc.
Finally, it writes the result of the analysis to the standard output or to an output file (if provided).

Pandalyzer uses the fork-join pattern to interpret both branches of an if-statement, where we are unable to derive which
branch would be taken and combines the results to a NondeterministicStructure.
The result of an invalid operation is the UnresolvedStructure, which ensures that the Pandalyzer is able to continue
with the interpretation even when an error in the code is discovered.
Pandalyzer gains information about the input csv file via a configuration file.
The configuration file also supports regex filenames.