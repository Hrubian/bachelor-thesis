\section{Example Case Study}

The Pandalyzer is able to find the mistakes done in the motivation example (Listing~\ref{lst:motivation_script}).
We also evaluated the Pandalyzer on five realistic case studies.
We show one of them.
The rest can be found in the Pandalyzer repository~\cite{pandalyzer}.

\definecolor{mygreen}{rgb}{0,0.6,0}
\definecolor{mygray}{rgb}{0.5,0.5,0.5}
\definecolor{mymauve}{rgb}{0.58,0,0.82}

\lstset{ %
    backgroundcolor=\color{white},   % choose the background color
    basicstyle=\footnotesize\ttfamily,        % size of fonts used for the code
    breaklines=false,                 % automatic line breaking only at whitespace
    captionpos=b,                    % sets the caption-position to bottom
    commentstyle=\color{mygreen},    % comment style
    escapeinside={\%*}{*)},          % if you want to add LaTeX within your code
    keywordstyle=\color{blue},       % keyword style
    stringstyle=\color{mymauve},     % string literal style
}


\lstinputlisting
[numbers=left, caption=Pandas code in Python, label={lst:code}, language=Python]
{splash_abstract/listings/case_study.py}

The listing~\ref{lst:code} shows an example of a Pandas code together with inline Dataframe structure definition
on lines 1, 3 and 6 (usually provided via a configuration file).
The Listing~\ref{lst:pandalyzer_output} contains the output of the Pandalyzer with information such as the variables
in the global scope (snipped in listing), errors, warnings and the structure of the output files.
This case study shows that the tool can handle many non-trivial operations.

\lstinputlisting
[caption=Pandalyzer output, label={lst:pandalyzer_output}]
{poster/listings/analysis_output.txt}
