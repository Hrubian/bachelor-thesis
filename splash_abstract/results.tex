\section{Conclusion}

Our work makes two major contributions.
The first contribution is the definition of the abstract domain for the abstract interpretation of Python code
that works with Pandas series and data frames.
The idea could be used with other languages like the Tibble~\cite{tibble} library in R\@.
The second result is the implementation of Pandalyzer---the Pandas analyzer based on the proposed framework.
The Pandalyzer is still in the early stage of the development process.
However, it already supports a large range of Pandas functions, and it proves that the idea is implementable and can be
very useful in practice.

The idea to use Abstract Interpretation to analyze programs working with Dataframes was already proposed
by Yungyu~Zhuang and~Ming-Yang~Lu~\cite{Zhuang:2022:TypeChecking}.
They also created a proof of concept implementation named PDChecker.
However, PDChecker does not have as many checks, does not report output CSV files (via to\_csv() function) and
does not allow for interpreting multiple branches of if-statements.
Also, Pandalyzer supports more Pandas functions and detects a wider range of errors.

The future works include adding support for other Pandas functions, other Python language constructs or including
Pandas Indexes in the analysis process.
The Pandalyzer could also be extended to support other well-known and related Python libraries such as Numpy,
Matplotlib or Scipy.

Another field where the Pandalyzer will be extended in the future is its integration to the IDEs such as PyCharm or
VS Code.
This could be done using the Language Server Protocol~\cite{language_server_protocol}.
We also intend to evaluate Pandalyzer empirically over a larger corpus of sample programs
(could be done using the CodeDJ~\cite{vakar2021causality}).