\chapter{Evaluation of solution}\label{ch:evaluation-of-solution}

This chapter is dedicated to the evaluation of the implementation presented in the previous chapter.
We design a set of realistic case studies for Pandas and evaluate the quality of analysis of the proposed solution.
We do not define any specific metrics for the evaluation since it is cumbersome to do it rigidly.
We rather look at the analysis from the more intuitive perspective and discuss which useful analysis features are
provided and which could be missing.

\section{Case Studies}

Each case study contains:
\begin{itemize}
    \item Explanation of the case
    \item An example code in Python
    \item Description of possible mistakes
    \item The output of the analyzer in the good case and in the case with mistakes
\end{itemize}

\paragraph{Case Study \#1:} Multiple operations, multiple Dataframes, no uncertainty  \\

The purpose of the first case study is to show the capabilities of Pandalyzer in a deterministic environment when
the Pandalyzer has all the information it needs for the analysis and there is no uncertainty.
There are, however, many different operations with multiple Dataframes.
We show that Pandalyzer is able to check input for all these operations and determine their output.

In this case study we work with a data from sport competitions agency.
We get a csv file with competitions that they organized.
We also get a csv file with all attendees of these competitions.
Last file contains information about the number of points that an attendee got from a specific competition.

The \verb|config.toml| (and the csv column structure) is shown in the listing~\ref{lst:cs1_config}

\begin{lstlisting}[caption=config.toml of the first case study, label={lst:cs1_config}, captionpos=b]
[attendees.csv]
name = "string"
surname = "string"
age = "int"

[matches.csv]
id = "int"
name = "string"

[scores.csv]
name_surname = "string"
match_id = "int"
score = "int"
\end{lstlisting}

Our goal is to create a csv file called top\_two\_per\_age.csv that contains the top two attendees per age per sport match
together with the sport match name.

The listing~\ref{lst:cs1_code} shows how that could be implemented in Pandas.

\begin{lstlisting}[caption=Solution of the first case study in Pandas, label={lst:cs1_code}, captionpos=b, numbers=left]
import pandas as pd

attendees_df = pd.read_csv("attendees.csv")
matches_df = pd.read_csv("matches.csv") \
    .rename(columns={"name": "match_name"})
scores_df = pd.read_csv("scores.csv")

attendees_df["name_surname"] = \
    attendees_df["name"] + "_" + attendees_df["surname"]
attendees_df = attendees_df.drop(columns=["name", "surname"])

scores_with_match_name_df = scores_df \
    .merge(matches_df, left_on="match_id", right_on="id") \
    .drop(columns="id")

scores_with_age_df = pd.merge(
    scores_with_match_name_df, attendees_df, on="name_surname"
)

top_two_per_age_df = scores_with_age_df \
    .sort_values("age") \
    .groupby(["age", "match_name"]) \
    .head(2) \
    .drop(columns=["match_id"])

top_two_per_age_df.to_csv("top_two_per_age.csv")
\end{lstlisting}

Let us break down the code in the listing~\ref{lst:cs1_code}.
We first read all the CSV files and store the data to Pandas Dataframes.
We also rename the \verb|name| column in \verb|matches_df| to \verb|match_name| so that the name of the column makes
sense later when it is merged with other Dataframes.
This is the first tricky part, since from now on we have to remember that the \verb|matches_df| does not contain a name
column but rather a \verb|match_name| column.
Then, since \verb|attendees_df| contains \verb|name| and \verb|surname| columns and the \verb|scores_df| contains
\verb|name_surname|, we also have to create a new column \verb|name_surname| in the \verb|attendees_df| and drop the
old columns (\verb|name| and \verb|surname|).
The next two operations merge the three Dataframes to a \verb|scores_with_age_df|.
We can notice that it is now already hard to keep track of the structure of the Dataframes.
Then the Dataframe is sorted by \verb|age|, grouped by \verb|age| and \verb|match_name| and the first two items from each group are selected.
Also, we drop the \verb|match_id| since we do not need it in the final data.
The resulting Dataframe is stored to a CSV file called \verb|top_two_per_age.csv|.

This case study shows two ideas.
First, it is hard to keep track of everything what is happening with the Dataframes, and it is easy to do a mistake
e.g.~access already non-existent column, misspell the column name or incorrectly specify the arguments of a pandas function.
Second, the scenario where we are able to deterministically derive the structure of the resulting data (e.g.~it does not
depend on any user input or other sources of uncertainty) can be a very realistic scenario. % todo is this clear enough?

What we want from the Pandalyzer in this case is to give us the structure of the \verb|top_two_per_age.csv| file
and warn us in case that we do any of the mistakes mentioned.
The listing~\ref{lst:cs1_pandalyzer_normal} shows the output of the Pandalyzer given the input shown in the
listing~\ref{lst:cs1_code} and the configuration shown in the listing~\ref{lst:cs1_config}.

\begin{lstlisting}[caption=Output of Pandalyzer on the first case study, label={lst:cs1_pandalyzer_normal}, captionpos=b]
Summary of analysis: OK
Global data structures (7):
pd: PandasImport
attendees_df: DataFrame(columns={age=IntType,
    name_surname=StringType})
matches_df: DataFrame(columns={id=IntType,
    match_name=StringType})
scores_df: DataFrame(columns={name_surname=StringType,
    match_id=IntType, score=IntType})
scores_with_match_name_df: DataFrame(columns={
    name_surname=StringType, match_id=IntType,
    score=IntType, match_name=StringType})
scores_with_age_df: DataFrame(columns={name_surname=StringType,
    match_id=IntType, score=IntType,
    match_name=StringType, age=IntType})
top_two_per_age_df: DataFrame(columns={name_surname=StringType,
    score=IntType, match_name=StringType, age=IntType})

Warnings (0):

Errors (0):

Output files (1):
File top_two_per_age.csv:
    name_surname : StringType
    score : IntType
    match_name : StringType
    age : IntType
\end{lstlisting}

The first line tells us that the no mistake was spotted in the analyzed code (otherwise there would be \verb|NOT OK|).
Then, the analyzer tells us all variables that are in the global scope when the analysis ended.
We can see all the Dataframes with their structure.
Next, we can see that the lists of Warnings and Errors are empty.
Finally, there is the \verb|Output files| summary with the \verb|top_two_per_age.csv| file structure.
The analyzer is able to tell us what columns the file contains as well as what are their types.

Now we show what happens when we do some mistake in the code.

The listing~\ref{lst:cs1_misspelled_column} shows the changed line 5 (\verb|neme| instead of \verb|name|) in the code as
well as the analysis output.

\begin{lstlisting}[caption=Misspelled column on rename function and analysis output, label={lst:cs1_misspelled_column}, captionpos=b]
    .rename(columns={"neme": "match_name"})

Summary of analysis: NOT OK
Global data structures (7):
pd: PandasImport
attendees_df: DataFrame(columns={age=IntType,
    name_surname=StringType})
matches_df: UnresolvedStructure(reason=The values
    [neme] do not exist in the dataframe)
scores_df: DataFrame(columns={name_surname=StringType,
    match_id=IntType, score=IntType})
scores_with_match_name_df: UnresolvedStructure(
    reason=Incorrect right argument to merge)
scores_with_age_df: UnresolvedStructure(reason=
    Incorrect left argument to merge)
top_two_per_age_df: UnresolvedStructure(reason=
    the attribute sort_values of UnresolvedStructure
    does not exist)

Warnings (0):

Errors (5):
0: Assign from line 4 to line 5 columns 0 - 44:
    The columns [neme] do not exist in the dataframe
1: Assign from line 12 to line 14 columns 0 - 23:
    Incorrect right argument to merge
2: Assign from line 16 to line 18 columns 0 - 1:
    Incorrect left argument to merge
3: Assign from line 20 to line 24 columns 0 - 31:
    the attribute sort_values of UnresolvedStructure
    does not exist
4: ExpressionStatement on line 26 columns 0 - 48:
    the attribute to_csv of UnresolvedStructure does not exist

Output files (0):
\end{lstlisting}

Now we can see that the result of the analysis is \verb|NOT OK|.
There are five errors in the Errors list.
The first error tells us that in the statement on lines 4--5 there is a statement where we are trying to access column
\verb|neme| that does not exist in the Dataframe.
This is enough for us to spot and fix the mistake in the code.
Other errors are in this case just consequences of the first error.

Other mistake that we can do is to specify an incorrect argument to a Pandas function.
The listing~\ref{lst:cs1_incorrect_argument} shows the changed part of the code on the line 13 (\verb|left_or| instead of
\verb|left_on|) and the analysis output.
Note that only important parts of the analysis output are shown.

\begin{lstlisting}[caption=Incorrectly specified argument and analysis output, label={lst:cs1_incorrect_argument}, captionpos=b]
    .merge(matches_df, left_or="match_id", right_on="id") \

Summary of analysis: NOT OK
/* snip */
Errors (4):
0: Assign from line 12 to line 14 columns 0 - 23:
    Got unexpected keyword arguments [left_or]
/* snip */
\end{lstlisting}

The result of the analysis is \verb|NOT OK| as expected and there is an error saying that there was an unexpected
keyword argument \verb|left_or| on the lines 12--14.


\paragraph{Case Study \#2:} Uncertainty from the user, multiple possible values \\

Next case study shows how the Pandalyzer behaves when there is some uncertainty such as an input from the user.
It also proves that the Pandalyzer is able to handle non-trivial control flow such as if-statements, functions and
early returns.

The code for this case study can be seen in the listing~\ref{lst:cs2_code}.

\begin{lstlisting}[caption=Code of the second case study in Pandas, label={lst:cs2_code}, captionpos=b]
import pandas as pd


def get_country_dataframe(country):
    if country == "Germany":
        return pd.read_csv("de.csv")
    elif country == "Austria":
        return pd.read_csv("au.csv")
    else:
        return pd.read_csv("world.csv")


def get_dataframe_from_user():
    country = input("Select a country: ")
    return get_country_dataframe(country)


user_df = get_dataframe_from_user()
user_df[["germany_specific_column"]].to_csv("output.csv")
\end{lstlisting}

And the configuration file is shown in the listing~\ref{lst:cs2_config}.

\begin{lstlisting}[caption=config.toml file for the second case study, label={lst:cs2_config}, captionpos=b]
[de.csv]
germany_specific_column = "string"
common_column = "int"

[au.csv]
common_column = "int"

[world.csv]
common_column = "int"
\end{lstlisting}

There are two functions.
The first (\verb|get_country_dataframe|) returns the Dataframe for the specified country or the Dataframe for the whole
world if the country is not known.
The second function (\verb|get_dataframe_from_user|) gets the country from the user and returns the result of
\verb|get_country_dataframe| function.
However, we do not know the result of the \verb|input| function, so we are not able to resolve which Dataframe will
be returned.
Moreover, we are trying to access the column \verb|germany_specific_column| which exists only if the user inputs
the value \verb|Germany|.
So the program can (but does not have to) crash.

The listing~\ref{lst:cs2_analysis_output} shows the output of the Pandalyzer given the input from the
listing~\ref{lst:cs2_code} and the configuration file from listing~\ref{lst:cs2_config}.

\begin{lstlisting}[caption=Analysis output of the second case study, label={lst:cs2_analysis_output}, captionpos=b]
Summary of analysis: OK
/* snip */
Warnings (6):
0: Assign on line 14 columns 4 - 41:
    Unable to resolve result of input
1: IfStatement from line 5 to line 10 columns 4 - 39:
    Unable to check unknown strings on equality
2: IfStatement from line 5 to line 10 columns 4 - 39:
    Unable to recognize the bool value in the
    if statement test - branching.
3: IfStatement from line 7 to line 10 columns 4 - 39:
    Unable to check unknown strings on equality
4: IfStatement from line 7 to line 10 columns 4 - 39:
    Unable to recognize the bool value in the
    if statement test - branching.
5: ExpressionStatement on line 19 columns 0 - 57:
    Second branch of execution failed with reason:
        Both execution branches failed.
        Branch1:
            The keys [germany_specific_column] do not
            exist in the dataframe,
        Branch2:
            The keys [germany_specific_column] do not
            exist in the dataframe

Errors (0):

Output files (1):
File output.csv:
    germany_specific_column : StringType

\end{lstlisting}

This time the output is more complicated.
The result of the analysis is \verb|OK| since the fact that some branch can fail does not mean that the whole program is
incorrect.
However, there are five warnings.
The first warnings is related to the \verb|input| function itself.
It just tells the user that an uncertainty occurred.
The next four warnings are related to the fact that we are trying to compare a string with unknown value and then
trying to branch based on the result.
The Pandalyzer tells us that it is branching (executing both branches) in that case.
The last warning is the most important.
It is less readable, but that is just because the problem is complicated.
It tells us that one of the branches failed, and it gives us the reason.
The reason is that both branches in the second branch failed, and it gives us the reasons.
The Pandalyzer also tells us that there is an output file.
It corresponds to the branch of execution in which the user input was \verb|Germany|.


\paragraph{Case Study \#3:} Regex config, column compatibility \\

Now we have a set of files 30\_04\_2024\_production.csv and 31\_04\_2024\_production.csv that contain per-hour production
of some factory on a given day.
We are interested in hours when the production was lower than 400 items.
This time we use the regex feature of our configuration.
The \verb|config.toml| file will be as shown in listing~\ref{lst:cs3_config}.

\begin{lstlisting}[caption=config.toml of the second case study, label={lst:cs3_config}, captionpos=b]
r[^\d{2}_\d{2}_\d{4}_production\.csv$]
hour = "int"
production = "int"
note = "string"
\end{lstlisting}

The record in the \verb|config.toml| file says that all the csv files with the name dd\_mm\_yyyy\_production.csv have the
specified columns structure.
The regex feature is useful for large amount of same-structured files with similar names (different only in date,
number etc.).

The code solving this problem can be seen in Listing~\ref{lst:cs3_code}

\begin{lstlisting}[caption=Solution of the third case study in Pandas, label={lst:cs3_code}, captionpos=b, numbers=left]
import pandas as pd

tuesday_df = pd.read_csv("30_04_2024_production.csv")
wednesday_df = pd.read_csv("31_04_2024_production.csv")

tuesday_df["day"] = 30
wednesday_df["day"] = 31

agg_df = pd.concat([tuesday_df, wednesday_df])

low_production_df = agg_df[agg_df["production"] < 400]

low_production_df.to_csv("aggregate_production.csv")
\end{lstlisting}

The output of the Pandalyzer given the input from listing~\ref{lst:cs3_code} and the configuration from
listing~\ref{lst:cs3_config} is shown in the listing~\ref{lst:cs3_analysis_output}

\begin{lstlisting}[caption=Analysis output of the third case study, label={lst:cs3_analysis_output}, captionpos=b]
Summary of analysis: OK
Global data structures (5):
pd: PandasImport
tuesday_df: DataFrame(columns={hour=IntType,
    production=IntType, note=StringType, day=IntType})
wednesday_df: DataFrame(columns={hour=IntType,
    production=IntType, note=StringType, day=IntType})
agg_df: DataFrame(columns={hour=IntType,
    production=IntType, note=StringType, day=IntType})
low_production_df: DataFrame(columns={hour=IntType,
    production=IntType, note=StringType, day=IntType})

Warnings (0):

Errors (0):

Output files (1):
File aggregate_production.csv:
    hour : IntType
    production : IntType
    note : StringType
    day : IntType
\end{lstlisting}

As expected, the result of the analysis is \verb|OK| and the lists of Warnings and Errors are empty.
There is one output file \verb|aggregate_production.csv| with the correct structure.
The Pandalyzer was also able to match the input filenames with the regex in the configuration file.
An important part of the analysis is the \verb|concat| function.
The result of the function has the same structure as the inputs, but it requires the inputs to have the same structure.
Let us see what the analyzer outputs if we change the structure of one of the Dataframes.
The listing~\ref{lst:cs3_incompatible_concat} shows the changed line TODO\ldots

\begin{lstlisting}[caption=Incompatible Dataframes to concat operation and analysis output, label={lst:cs3_incompatible_concat}, captionpos=b]
tuesday_df["another_column"] = 30

Summary of analysis: NOT OK
/* snip */
Errors (3):
0: Assign on line 9 columns 0 - 46:
    All dataframes to be concatenated must have
    the same column structure
/* snip */
\end{lstlisting}

The Pandalyzer detected the mistake.
The result of the analysis is \verb|NOT OK| and there is an error saying that all Dataframes ot be concatenated must
have the same structure.


\paragraph{Case Study \#4:} High uncertainty, user input \\

The fourth case study again focuses on the uncertainty, but it shows different use cases of the uncertainty than the
second case study.

This time, we do not have any configuration file.
There is just a source code that is shown in the listing~\ref{lst:cs4_code}.

\begin{lstlisting}[caption=Code for the fourth case study in Pandas, label={lst:cs4_code}, captionpos=b, numbers=left]
import pandas as pd

df = pd.DataFrame({
    "string_column": [
        input("First string: "),
        input("Second string: "),
        input("Third string: ")
    ],
    "int_column": [
        int(input("First int: ")),
        int(input("Second int: ")),
        int(input("Third int: "))
    ]
})

print(df[input("What column do you want to see? ")])

df["note"] = "User inserted the following string" \
             + df["string_column"]

df.to_csv("output.csv")
\end{lstlisting}

Even though the whole Dataframe \verb|df| is constructed from the user input, the Pandalyzer should have enough
information to derive its structure.
The Pandalyzer should also be able to derive the type of the \verb|note| column.

The listing~\ref{lst:cs4_analysis_output} shows the actual output of the analysis.

\begin{lstlisting}[caption=Analysis output of the fourth case study, label={lst:cs4_analysis_output}, captionpos=b]
Summary of analysis: OK
Global data structures (2):
pd: PandasImport
df: DataFrame(columns={string_column=StringType,
    int_column=IntType, note=StringType})

Warnings (11):
0: Assign from line 3 to line 14 columns 0 - 2:
    Unable to resolve result of input
1: Assign from line 3 to line 14 columns 0 - 2:
    Unable to resolve result of input
2: Assign from line 3 to line 14 columns 0 - 2:
    Unable to resolve result of input
3: Assign from line 3 to line 14 columns 0 - 2:
    Unable to determine the string value for int function
4: Assign from line 3 to line 14 columns 0 - 2:
    Unable to resolve result of input
5: Assign from line 3 to line 14 columns 0 - 2:
    Unable to determine the string value for int function
6: Assign from line 3 to line 14 columns 0 - 2:
    Unable to resolve result of input
7: Assign from line 3 to line 14 columns 0 - 2:
    Unable to determine the string value for int function
8: Assign from line 3 to line 14 columns 0 - 2:
    Unable to resolve result of input
9: ExpressionStatement on line 16 columns 0 - 52:
    The key for subscript of dataframe is not known
10: ExpressionStatement on line 16 columns 0 - 52:
    Unable to resolve result of input

Errors (0):

Output files (1):
File output.csv:
    string_column : StringType
    int_column : IntType
    note : StringType
\end{lstlisting}

The first thing to notice is that the result of the analysis is \verb|OK|.
Another important information is that the Pandalyzer was actually able to derive the type of the \verb|output.csv| file.
The Pandalyzer knows that the \verb|input| function returns string and the \verb|int| function returns an int if
it receives a string as an input.
It does not need to know the exact values to construct the Dataframe structure.
At last, there are eleven warnings.
All of them are caused by the uncertainty and are mostly related to the \verb|input| and \verb|int| functions.


\paragraph{Case Study \#5:} Group by, aggregations \\

In the last case study, we go over some use cases of the group by operation and the associated aggregations.
Let us have the following configuration file (shown in listing~\ref{lst:cs5_config}):

\begin{lstlisting}[caption=config.toml file for the fifth case study, label={lst:cs5_config}, captionpos=b]
[all_strings.csv]
str1 = "string"
str2 = "string"
str3 = "string"

[first_string.csv]
col1 = "string"
col2 = "int"
col3 = "int"

[first_int.csv]
col1 = "int"
col2 = "string"
col3 = "string"

[all_different.csv]
int_col = "int"
str_col = "string"
bool_col = "bool"
\end{lstlisting}

The source code for the case study can be seen in the listing~\ref{lst:cs5_code}.

\begin{lstlisting}[caption=Code of the fifth case study in Pandas, label={lst:cs5_code}, captionpos=b, numbers=left]
import pandas as pd

all_strings_df = pd.read_csv("all_strings.csv")
first_string_df = pd.read_csv("first_string.csv")
first_int_df = pd.read_csv("first_int.csv")
all_different_df = pd.read_csv("all_different.csv")

pass_df1 = first_string_df.groupby("col1").mean()
fail_df1 = first_string_df.groupby("col2").mean()

pass_df2 = all_strings_df.groupby("str1").count()["str2"] + 3
fail_df2 = all_strings_df.groupby("str1").count()["str2"] + "hi"

pass_df3 = all_different_df.groupby("bool_col").sum()
fail_df3 = all_different_df.groupby("str_col").sum()

pass_df4 = first_int_df.groupby(["col2", "col3"]).mean()
fail_df4 = first_int_df.groupby("col3").mean()
\end{lstlisting}

The code reads all four CSV files defined and stores them in the Dataframes.
Then, there are four pairs group by operations followed by aggregations.
Every pair has one operation that passes and one that fails.

The first pair applies the \verb|mean| function on the \verb|first_string_df| grouped by \verb|col1| and by \verb|col2|.
However, only the first option passes.
The reason is that the mean function is applied to all columns that we did not group by.
So in the second case it is also the \verb|col1| column which is string.
But the mean function cannot be applied on a string column.

The second pair shows the usage of \verb|count| function on the \verb|all_strings_df| grouped by \verb|str1|.
The count should return an int column for all columns that we did not group by.
So summing some of the columns with a number should pass and summing it with string should fail.

In the third pair we apply the sum function to non-grouped columns.
In the first case it is \verb|str_col| and \verb|int_col|, which can be summed.
In the second case it is \verb|int_col| and \verb|bool_col|.
The second case will actually \textbf{not} fail this time.
But summing of bool column is usually something that we do not want, so we should consider it an error.

The last (fourth) case again computes a mean.
This time, the first case will not fail, since the only column we apply mean on is \verb|col1| which is an int column.
The second case will fail, since we apply the mean on both the \verb|col1| and \verb|col2|, but the \verb|col2| is
a string column which we cannot apply mean on.

The listing~\ref{lst:cs5_analysis_output} shows the output of the Pandalyzer on the code in the listing~\ref{lst:cs5_code}
and the configuration in the listing~\ref{lst:cs5_config}.

\begin{lstlisting}[caption=Analysis output of the fifth case study, label={lst:cs5_analysis_output}, captionpos=b]
Summary of analysis: NOT OK
\* snip *\
Warnings (0):

Errors (4):
0: Assign on line 9 columns 0 - 49:
    Cannot apply mean on the columns: col1 of type StringType,
1: Assign on line 12 columns 0 - 67:
    Cannot sum a series of type IntType with PythonString
2: Assign on line 15 columns 0 - 52:
    Cannot apply sum on the columns: bool_col of type BoolType,
3: Assign on line 18 columns 0 - 46:
    Cannot apply mean on the columns: col2 of type StringType,

Output files (0):
\end{lstlisting}

The analysis shows exactly what we said.
There are four errors, each explaining one of the mistakes discussed.

\section*{Summary}
\addcontentsline{toc}{section}{Summary}

The Pandalyzer is already capable of checking many various Pandas operations.
It is also able to work with some degree of uncertainty as seen in the case studies.
In each analysis, the Pandalyzer tells us the analysis result (\verb|OK| or \verb|NOT OK|), active variables in the
scope, list of warnings, errors and all output files created together with their structure.
The errors and warnings contain information about the line and column of the source statement, type of the statement
and some message describing the issue.
However, to be actually useful in practice, there is a need to implement other Pandas operations and Python constructs.
The future of the Pandalyzer will be discussed in the Future Work~\ref{sec:future-work} section of the Conclusion
