\chapter{Evaluation of solution}

This chapter is dedicated to the evaluation of the implementation presented in the previous chapter.
We design a set of realistic case studies for Pandas and evaluate the quality of analysis of the proposed solution.
We do not define any specific metrics for the evaluation since it is cumbersome to do it rigidly.
We rather look at the analysis from the more intuitive perspective and discuss which useful analysis features are
provided and which could be missing.

\section{Case Studies}

Each case study contains:
\begin{itemize}
    \item Explanation of the case
    \item An example code in Python
    \item Description of possible mistakes
    \item The output of the analyzer in the good case and in the case with mistakes
\end{itemize}

\paragraph{Case Study \#1:} Multiple operations, multiple datasets, no uncertainty  \\

In the first case study we work with a data from sport competitions agency.
We get a csv file with competitions that they organized.
We also get a csv file with all attendees of these competitions.
Last file contains information about the number of points that an attendee got from a specific competition.

The \verb|config.toml| (and the csv column structure) is shown in the listing~\ref{lst:cs1_config}

\begin{lstlisting}[caption=config.toml of the first case study, label={lst:cs1_config}, captionpos=b]
[attendees.csv]
name = "string"
surname = "string"
age = "int"

[matches.csv]
id = "int"
name = "string"

[scores.csv]
name_surname = "string"
match_id = "int"
score = "int"
\end{lstlisting}

Our goal is to create a csv file called top\_two\_per\_age.csv that contains the top two attendees per age per sport match
together with the sport match name.

The listing~\ref{lst:cs1_code} shows how that could be implemented in Pandas.

\begin{lstlisting}[caption=Solution of the first case study in Pandas, label={lst:cs1_code}, captionpos=b]
import pandas as pd

attendees_df = pd.read_csv("attendees.csv")
matches_df = pd.read_csv("matches.csv") \
    .rename(columns={"name": "match_name"})
scores_df = pd.read_csv("scores.csv")

attendees_df["name_surname"] = \
    attendees_df["name"] + "_" + attendees_df["surname"]
attendees_df = attendees_df.drop(columns=["name", "surname"])

scores_with_match_name_df = scores_df \
    .merge(matches_df, left_on="match_id", right_on="id") \
    .drop(columns="id")

scores_with_age_df = pd.merge(
    scores_with_match_name_df, attendees_df, on="name_surname"
)

top_two_per_age_df = scores_with_age_df \
    .sort_values("age") \
    .groupby(["age", "match_name"]) \
    .head(2) \
    .drop(columns=["match_id"])

top_two_per_age_df.to_csv("top_two_per_age.csv")
\end{lstlisting}

\xxx{TODO what is tricky about this example, what does the analyzer say, what mistakes can we do, does analyzer spot them? how?}


\paragraph{Case Study \#2:} Uncertainty from the user, multiple possible values \\

Next case study.


\paragraph{Case Study \#3:} Regex config, column compatibility \\

Now we have a set of files 30\_04\_2024\_production.csv and 31\_04\_2024\_production.csv that contain per-hour production
of some factory on a given day.
We are interested in hours when the production was lower than 400 items.
This time we use the regex feature of our configuration.
The \verb|config.toml| file will be as shown in listing~\ref{lst:cs3_config}.

\begin{lstlisting}[caption=config.toml of the second case study, label={lst:cs3_config}, captionpos=b]
[^\d{2}_\d{2}_\d{4}_production\.csv$]
hour = "int"
production = "int"
note = "string"
\end{lstlisting}

The record in the \verb|config.toml| file says that all the csv files with the name dd\_mm\_yyyy\_production.csv have the
specified columns structure.
The regex feature is useful for large amount of same-structured files with similar names (different only in date,
number etc.).

The code solving this problem can be seen in Listing~\ref{lst:cs3_code}

\begin{lstlisting}[caption=Solution of the third case study in Pandas, label={lst:cs3_code}, captionpos=b]
import pandas as pd

tuesday_df = pd.read_csv("30_04_2024_production.csv")
wednesday_df = pd.read_csv("31_04_2024_production.csv")

tuesday_df.insert(0, "day", 30)
wednesday_df.insert(0, "day", 31)

agg_df = pd.concat([tuesday_df, wednesday_df])

low_production_df = agg_df[agg_df["production"] < 400]

low_production_df.to_csv("aggregate_production.csv")
\end{lstlisting}

\paragraph{Case Study \#4:} High uncertainty, user input \\
This time we focus on


\section*{Summary}
\addcontentsline{toc}{section}{Summary}