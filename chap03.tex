\chapter{Putting it all together} % TODO maybe change the name later :)

% TODO change 'As the name of this chapter' if you change the name of the chapter
As the name of this chapter suggests, now we put the knowledge from the previous chapters together.
We build the Abstract Interpretation framework for data manipulation programs.
It involves defining the concrete and abstract lattice and the galois connection between.

Since we are mostly focusing on Pandas in Python, we assume Python environment with Pandas, specifically:

\begin{itemize}
    \item Python syntax
    \item Standard Python data types - int, float, string, list, dictionary, tuple
    \item Pandas is imported via \verb|import pandas as pd| statement
\end{itemize}

We also assume, unlike the Pandas, that Series are homogenous and Dataframes have homogenous columns.
This also implies that each column of a Dataframe has an associated type.
Let us formalize this: 

\begin{defn}

\end{defn}

Given these assumptions, we can define a Dataframe Structure:

\begin{defn}[Dataframe Structure]
    For a Dataframe $Df$, its \textbf{Dataframe} Structure is $DS(Df) = \{(Col.name, Col.type) \: for \: each \: column \: in \: Df\}$
\end{defn}

Also, we define a Series Structure:

\begin{defn}[Series Structure]
    For a Series $Sr$, its \textbf{Series Structure} is $SS(Sr) = (Sr.name, Sr.type)$
\end{defn}


\section{The concrete lattice}

The concrete lattice should correspond with the reality.


\section{The abstract lattice}

\section{The Galois Connection}

\section{Operations}

\section{Adding other types}

\section{Final proposal}

\section{Limitations}


\section*{Summary}
\addcontentsline{toc}{section}{Summary}