\chapter*{Introduction}
\addcontentsline{toc}{chapter}{Introduction}

An important part of~data analysis is~called data~manipulation or data~wrangling.
The~goal of~data~manipulation is~to~take the raw~data obtained from~some source, clean them and transform them into
better-structured error-free data which are more suitable for the~data~analysis itself.
The data manipulations -~cleaning~and~transformations~- usually cannot be~done manually, since the~data are too large
for~that.
For manipulating large amounts of data, one needs an~automated tool.
Popular approach to solve this problem is a~library in~a~programming language.
Data~manipulation libraries that became widely used in~the~industry, are Pandas~in~Python\cite{pandas_docs},
Tibble~in~R\cite{tibble} (R also has built-in
support for~data~frames\cite{R-base}) or DataFrames.jl~in~Julia\cite{DataFrames.jl}.
These libraries usually come with two types of data structures: one-dimensional array usually called Series
and two-dimensional tabular structure usually called Dataset~or~Dataframe.

The~mentioned languages, Python, R and Julia, are~all~dynamic and~interpreted languages, meaning that they do~not~provide
the~programmer with any~compile-time checks for type consistency and~variable or~column existence.
These languages are~popular in~data-related subjects due~to~their easy~syntax and~the~fact that
one does~not need to~complete a course on~computer architecture to~use~it.
However, the user of~a~dynamic languages has~to~keep most of~the~information in~their~head (or~use comments~in~code a~lot)
which is~a~potential source of~many different errors, that are, in~addition, detected no~sooner than at~runtime when
the program crashes.
Consider for example the program in Listing~\ref{lst:pandas_errors}.
There are some harder-to-spot mistakes such as referencing a dropped column, summing columns of different types
or a misspelled column name.
All these mistakes are detected at~runtime causing crash of the program.

\begin{lstlisting}[caption=Pandas code with errors, label={lst:pandas_errors}, captionpos=b]
import pandas as pd

df = pd.read_csv("data.csv")
df_copy = df
df_copy.drop("column1", inplace=True)

grouped = df.groupby("column1")
# Error - column1 does not exist already

final_score = df["score_a"] + df["score_b_note"]
# Error - summing series of ints with strings

print(df["colunm2"])
# Error - misspelled column name colunm2

\end{lstlisting}

A potential approach to prevent these issues could~be to~use a~different programming language.
One~with a~static strict~type~system, that would allow~us to~use typed data-frames~-~a data frames with statically~assigned
column~names and~types.
The results of data-frame transformations would then be determined by the function type signature.
Even though this~solution would solve the~issue with~column~naming and column~types, it~has many problems that dynamic
languages do~not have.
The~arguments mentioned for~the~popularity of~dynamic languages are~also~arguments against static languages.
Namely, a sharp~learning~curve, a complicated syntax, and a~technical complexity.
Additionally, the~necessity for~a~lot~of~boilerplate code, worse readability and~the~fact, that~dynamic languages,
contrary~to the~static languages, usually support a REPL~tool or~just creating one~file and running~it.
Ideally, we would~like to~use dynamic languages like~Python, R or Julia, while keeping the~safety of~static languages.

\section*{Thesis goals}
\textbf{The~goal~of~this~thesis}~is to~propose and~explore an~alternative solution to~the~problem of~data-manipulation
code being error-prone and~provide an~implementation of~such~solution.
The~solution will~be~based on~a~program verification method called Abstract Interpretation.
The origins of~the~ideas behind the~method can~be~traced back~to~the~1970s when it~was~developed by~Patrick
and~Radhia~Cousot\cite{Cousot:1977:AI}.

In~essence, Abstract Interpretation partially executes the~given computer program over~an~abstract domain
(such as \{+, 0, -\} for integers).
It gains partial~information about the~programs semantics and~answers questions (as~mentioned
by~B.~Blanchet\cite{Blanchet:2002:AI}) such~as:
\begin{itemize}
    \item What is the worst-case execution time?
    \item Is the program secure against a specific attack?
    \item How much stack will the program use?
    \item Which parts of the code are unreachable?
\end{itemize}

The~idea of~analyzing programs working with~dataframes was~already proposed by~Yungyu~Zhuang
and~Ming-Yang~Lu~\cite{Zhuang:2022:TypeChecking}.
However, we~aim to~present more~fundamental analysis and develop a more comprehensive tool.
We will define our own framework with more levels of abstraction and expand the abstraction to non-pandas types as well.
Also, we will interpret both branches of if-statements in case that we do not have enough information to choose one, and
we plan to be able to work with more uncertainty from the user.

In~the~first chapter of~the~thesis, we~will review the~theory of~POSETs, Lattices and~Galois~connection which
are~the~core mathematical concepts of~Abstract Interpretation.
Then, in~the~second chapter, we~will explore the~needs of~data scientists regarding the~tools for~data~manipulation.
We~will~also take~a~look at~how Pandas provide the data~scientists with these needs.
The~third chapter will~put all~the~pieces together.
We~will define an~Abstract Interpretation framework for~the~world of~data-frames and~series.
We~will also take~a~look at~how some basic operations like~merge, groupby or concatenation will~work in~the~abstract domain.
The~fourth chapter will~be~dedicated to~the~implementation of~the~analysis~tool itself, its~architecture, documentation,
capabilities, and~limitations.
Finally, in~the~last, fifth, chapter, we~will evaluate the implemented solution on~a~set of~realistic examples of~Pandas
code and~analyze and~discuss its~usability in~practice.