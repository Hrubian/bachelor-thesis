\documentclass{tikzposter}
\geometry{paperwidth=842mm,paperheight=1185mm}
\usepackage{listings}

% Title, Author, Institute
\title{Abstract Interpretation of Pandas}
\author{Jan Hrubý}
\institute{Charles University, Faculty of Mathematics and Physics}
%\titlegraphic{LogoGraphic Inserted Here}

%Choose Layout
\usetheme{Simple}

\begin{document}

    \maketitle

    \begin{columns}

        % FIRST column
        \column{0.333}

        \block{Motivation}
        {
            \Large
            Pandas is a Python library widely used for data manipulation. %todo
            The code written with Pandas lacks any type-safety and everything is decided at runtime.
            This can potentially be a source of errors and crashes at runtime.
            Consider for example the following program:
            \lstinputlisting{lst_motivation.py}
            TODO
        }

        \block{Abstract Interpretation}
        {
            \Large
            Abstract Interpretation is a program analysis method based on the idea
            of abstracting the semantics of a program and interpretation of the
            program over the abstracted semantics.
            It uses the theory of Lattices and Galois Connection and is a sound method.
        }

        % SECOND column
        \column{0.333}

        \block{Framework}
        {
            \Large
            The Pandas library provides the user with two main data structures: one-dimensional Series
            and two-dimensional tabular DataFrame.
            To be able to interpret the program with Pandas over the abstract domain, we must define the abstract lattice.
            The picture TODO shows how the abstract lattice is defined.
            There is the UnresolvedStructure at the bottom of the hierarchy, representing a value that we were not
            able to derive(usually due to some error in execution).
        }

        \block{Implementation}
        {
            \Large
            \begin{itemize}
                \item Implemented in Kotlin
                \item command-line interface
                \item configuration file with input file structure and regex support
                \item supports subset of Python syntax
                \item gives information about the errors in the code, output file structure, global variables and useful warnings
                \item supports human-readable and JSON formats
                \item uses Python ast library for the source code parsing and AST creation
                \item supports unknown structures as a result of user input
                \item interprets multiple branches of the program in case where unable to choose the right branch
                \item continues with interpretation even when an error occurs
            \end{itemize} % todo capital letters?
        }

        % THIRD column
        \column{0.333}

        \block{Capabilities}

        \block{Conclusion}
        {
            \Large
            Our work has two major results.
            The first result is the proposition of the framework for abstract interpretation of data-manipulation programs.
            The idea of the framework could be used with other data-manipulation libraries in other languages like
            Tibble in R (R also has built-in support for data frames) or DataFrames.jl in Julia.
            The second result is the implementation of the Pandalyzer - Pandas analyzer based
            on the Abstract Interpretation framework proposed.
            The Pandalyzer is still in the early stage of the development process and a lot of work is planned.
            It serves as a proof of concept of the Abstract Interpretation framework for data-manipulation programs.
            However, it shows that the idea is implementable and can be very useful in practice.
        }

    \end{columns}



    \begin{columns}
        \column{0.5}

        \block{Acknowledgements} {meow}

        \column{0.5}

        \block{Further info} {meow}

    \end{columns}

\end{document}