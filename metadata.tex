%%% Please fill in basic information on your thesis, which will be automatically
%%% inserted at the right places.

% Type of your thesis:
%	"bc" for Bachelor's
%	"mgr" for Master's
%	"phd" for PhD
%	"rig" for rigorosum
\def\ThesisType{bc}

% Language of your study programme:
%	"cs" for Czech
%	"en" for English
\def\StudyLanguage{cs}

% Thesis title in English (exactly as in the official assignment)
% (Note: \xxx is a "ToDo label" which makes the unfilled visible. Remove it.)
\def\ThesisTitle{\xxx{Thesis title}}

% Author of the thesis (you)
\def\ThesisAuthor{\xxx{Name Surname}}

% Year when the thesis is submitted
\def\YearSubmitted{\xxx{YEAR}}

% Name of the department or institute, where the work was officially assigned
% (according to the Organizational Structure of MFF UK in English,
% see https://www.mff.cuni.cz/en/faculty/organizational-structure,
% or a full name of a department outside MFF)
\def\Department{\xxx{Name of the department}}

% Is it a department (katedra), or an institute (ústav)?
\def\DeptType{\xxx{Department}}

% Thesis supervisor: name, surname and titles
\def\Supervisor{\xxx{Supername Supersurname}}

% Supervisor's department (again according to Organizational structure of MFF)
\def\SupervisorsDepartment{\xxx{department}}

% Study programme (does not apply to rigorosum theses)
\def\StudyProgramme{\xxx{study programme}}

% An optional dedication: you can thank whomever you wish (your supervisor,
% consultant, who provided you with tea and pizza, etc.)
\def\Dedication{%
\xxx{Dedication.}
}

% Abstract (recommended length around 80-200 words; this is not a copy of your thesis assignment!)
\def\Abstract{%
\xxx{TODO finish english abstract}
Abstract interpretation is an approximation method for semantics of computer programs.
The method is widely used in compilers for finding optimization opportunities or as a
certification against certain types of bugs.
It systematically interprets the program semantics over an abstract domain which is
connected with the original domain by a galois connection.


The method could be also useful for analysis of data-analysis programs, where the most
popular languages are dynamic languages python and R. The resulting programs are
usually error-prone, since


The goal of this paper is to explore the possibility to use abstract interpretation
to analyse data-analysis programs working with tabular data and create a PoC
implementation of the idea for pandas library in python.
}

% 3 to 5 keywords (recommended) separated by \sep
% Keywords are useful for indexing and searching for the theses by topic.
\def\ThesisKeywords{%
\xxx{keyword\sep key phrase}
python
\sep
pandas
\sep
abstract interpretation
\sep
data analysis
}

% If any of your metadata strings contains TeX macros, you need to provide
% a plain-text version for use in XMP metadata embedded in the output PDF file.
% If you are not sure, check the generated thesis.xmpdata file.
\def\ThesisAuthorXMP{\ThesisAuthor}
\def\ThesisTitleXMP{\ThesisTitle}
\def\ThesisKeywordsXMP{\ThesisKeywords}
\def\AbstractXMP{\Abstract}

% If your abstracts are long and do not fit in the infopage, you can make the
% fonts a bit smaller by this setting. (Also, you should try to compress your abstract more.)
\def\InfoPageFont{}
%\def\InfoPageFont{\small}  % uncomment to decrease font size

% If you are studing in a Czech programme, you also need to provide metadata in Czech:
% (in English programmes, this is not used anywhere)

\def\ThesisTitleCS{\xxx{Název práce česky}}
\def\DepartmentCS{\xxx{Název katedry česky}}
\def\DeptTypeCS{\xxx{Katedra}}
\def\SupervisorsDepartmentCS{\xxx{katedra vedoucího}}
\def\StudyProgrammeCS{\xxx{studijní program}}

\def\ThesisKeywordsCS{%
\xxx{klíčová slova\sep klíčové fráze}
}

\def\AbstractCS{%
\xxx{TODO finish czech abstract}
}
