%\small

\setlength{\columnsep}{20cm}
\begin{multicols}{2}[
The thesis contains five case studies demonstrating the practical capabilities of the Pandalyzer.
We show one of them here. The listing~\ref{lst:code} shows an example Pandas code in Python.
The listing~\ref{lst:config} contains the information about the input CSV files and needs to be provided to the
Pandalyzer via a config file.
The listing~\ref{lst:output} shows the output of the Pandalyzer after analyzing the code in listing~\ref{lst:code}.
]

\definecolor{mygreen}{rgb}{0,0.6,0}
\definecolor{mygray}{rgb}{0.5,0.5,0.5}
\definecolor{mymauve}{rgb}{0.58,0,0.82}

\lstset{ %
    backgroundcolor=\color{white},   % choose the background color
    basicstyle=\footnotesize,        % size of fonts used for the code
    breaklines=false,                 % automatic line breaking only at whitespace
    captionpos=b,                    % sets the caption-position to bottom
    commentstyle=\color{mygreen},    % comment style
    escapeinside={\%*}{*)},          % if you want to add LaTeX within your code
keywordstyle=\color{blue},       % keyword style
stringstyle=\color{mymauve},     % string literal style
}


\lstinputlisting
    [basicstyle=\normalsize\ttfamily, numbers=left, caption=Pandas code in Python, label={lst:code}, language=Python]
    {poster/listings/case_study.py}

% for a line break
\vfill\null
\columnbreak

\lstinputlisting
    [basicstyle=\normalsize\ttfamily, caption=Configuration file, label={lst:config}]
    {poster/listings/case_study_config.toml}
\lstinputlisting
    [basicstyle=\normalsize\ttfamily, caption=Output of Pandalyzer, label={lst:output}]
    {poster/listings/analysis_output.txt}

\end{multicols}